\documentclass[a4paper,12pt]{article}
\usepackage[utf8]{inputenc}
\usepackage[brazil]{babel}
\usepackage[T1]{fontenc}
\usepackage{lmodern}
\usepackage{hyperref}
\usepackage{graphicx}
\usepackage{amsmath}
\usepackage{geometry}
\usepackage{listings}
\usepackage{xcolor}

\lstdefinestyle{pythonstyle}{
    language=Python,
    basicstyle=\ttfamily\small,
    keywordstyle=\color{blue}\bfseries,
    stringstyle=\color{teal},
    commentstyle=\color{gray}\itshape,
    showstringspaces=false,
    numbers=left,
    numberstyle=\tiny\color{gray},
    stepnumber=1,
    numbersep=8pt,
    frame=single,
    rulecolor=\color{black!30},
    breaklines=true,
    breakatwhitespace=true,
    tabsize=4,
    captionpos=b
}
\geometry{margin=2.5cm}

\title{Detecção de Fraude em Transações \\ \large Relatório de Atividade}
\author{Seu Nome \\ Matrícula: 0000000 \\ Disciplina XYZ}
\date{\today}

\begin{document}
\maketitle
\thispagestyle{empty}

\newpage
\tableofcontents
\thispagestyle{empty}

\newpage
\pagenumbering{arabic}

%----------------------------------------
\section{Introdução}
\label{sec:introducao}
\begin{itemize}
  \item Contextualização do problema de detecção de fraude
  \item Objetivos do trabalho
  \item Descrição do dataset (fonte, número de amostras, desbalanceamento)
\end{itemize}

%----------------------------------------
\section{Metodologia (Parte Prática)}
\label{sec:metodologia}

\subsection{Análise Exploratório de Dados (EDA) e Pré-processamento}
\label{subsec:eda}
\begin{itemize}
  \item Estatísticas descritivas e visualização das distribuições
  \item Tratamento de valores ausentes, codificações e escalonamento
  \item Estratégia para lidar com desbalanceamento
\end{itemize}

\subsection{Abordagem 1: Modelagem da Classe Normal (Detecção de Anomalia)}
\label{subsec:abordagem1}
\begin{itemize}
  \item Escolha do modelo generativo probabilístico
  \item Estimação de parâmetros via MLE
  \item Cálculo do score de anomalia
  \item Visualização das distribuições de scores
  \item Seleção e justificativa do limiar
\end{itemize}

\subsection{Abordagem 2: Classificação Supervisionada}
\label{subsec:abordagem2}
\begin{itemize}
  \item Treinamento de modelo probabilístico
  \item Geração de probabilidades $p(C_{\text{fraude}}\mid x)$
  \item Visualização das distribuições de probabilidade
  \item Escolha do limiar e justificativa
\end{itemize}

\subsection{Validação e Generalização}
\label{subsec:validacao}
\begin{itemize}
  \item Estratégia de validação (cross-validation, hold-out)
  \item Seleção de modelo e otimização de hiperparâmetros
\end{itemize}

%----------------------------------------
\section{Resultados}
\label{sec:resultados}
\begin{itemize}
  \item Métricas independentes de limiar: AUC-PR
  \item Métricas dependentes de limiar: Precisão, Recall, F1, matriz de confusão
  \item Gráficos de curva PR e distribuição de scores/probabilidades
\end{itemize}

%----------------------------------------
\section{Discussão e Comparação Final}
\label{sec:discussao}
\begin{itemize}
  \item Comparação de performance entre as abordagens
  \item Vantagens, desvantagens e limitações
  \item Impacto prático da escolha de limiares
  \item Recomendações de uso
\end{itemize}

%----------------------------------------
\section{Conclusão}
\label{sec:conclusao}
\begin{itemize}
  \item Principais achados
  \item Sugestões de trabalhos futuros
\end{itemize}

%----------------------------------------
\section{Respostas Teórico-Conceituais (Parte 2)}
\label{sec:teoria}

\subsection{Exercício 1: Fronteira de decisão em modelos generativos}
\label{subsec:ex1}

\subsection{Exercício 2: Verossimilhança, entropia cruzada e regressão logística}
\label{subsec:ex2}

\subsection{Exercício 3: Decomposição viés-variância}
\label{subsec:ex3}

\subsection{Exercício 4: Avaliação em dados desbalanceados}
\label{subsec:ex4}

%----------------------------------------
\section{Referências}
\label{sec:referencias}
% Exemplo de uso de BibTeX:
% \bibliographystyle{ieeetr}
% \bibliography{referencias}

\begin{thebibliography}{9}
  \bibitem{scikit-learn}
    Pedregosa et al.,
    \emph{Scikit-learn: Machine Learning in Python},
    JMLR, 2011.
  % adicione outras referências conforme necessário
\end{thebibliography}

\end{document}
